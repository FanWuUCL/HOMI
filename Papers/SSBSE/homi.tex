\documentclass[oribibl]{llncs}
\usepackage{llncsdoc}
\usepackage{graphicx}
\usepackage[numbers]{natbib}

\begin{document}

\title{Searching Higher Order Mutation For Software Improvement}

\author{Fan Wu\inst{1}         \and
		Mark Harman\inst{1}        \and
		Yue Jia\inst{1}         \and
		Jens Krinke\inst{1}
}


\institute{Department of Computer Science, UCL, Gower Street, London WC1E 6BT, UK \\
              \email{\{fan.wu.12,mark.harman,yue.jia,j.krinke\}@ucl.ac.uk} 
}

\maketitle

\begin{abstract}
Automation has been applied to software improvement in many ways.
However, it suffers from the dilemma of scalability and granularity.
We propose to use equivalent Higher Order Mutants to improve multiple non-functional properties of a software while the functional behavior is preserved.
We find that: zeller numbers go here.
%\keywords{Genetic Improvement \and SBSE \and Higher Order Mutation}
\end{abstract}

\section{Introduction}
\label{sec_intro}

Optimising softwares for better performance such as speed and memory consumption can be demanded, especially when the resources in the running environment is limited.
Manually optimising the performance while keeping or even improving the functional behavior of a software is challenging, 
and is even harder if there are several considered properties competing with each other~\cite{Harman:2012:GCC:2351676.2351678}.
Previous studies have applied Genetic Programming and other techniques to automate this optimisation process~\cite{6035728, geneticimprovementJP, Wu:2015:DPO:2739480.2754648}.


\section{Background}
\label{sec_back}

\section{Method}
\label{sec_method}

\subsection{Sensitivity Analysis}
\label{sec_sensitivity}

\subsection{Higher Order Mutation}
\label{sec_hom}

\subsection{Research Questions}
\label{sec_rqs}
In this paper, we are interested in the following Research Questions.

\begin{enumerate}
\item[\emph{RQ1}] Can First Order Mutants improve performance while maintaining functionality?
\end{enumerate}



\begin{enumerate}
\item[\emph{RQ2}] How much can Higher Order Mutants improve the original and FOMs' performance?
 \begin{enumerate}
\item[\emph{RQ2.1}] How much improvement can be achieved by using traditional mutation operators?
\item[\emph{RQ2.2}] How much improvement is accounted for by Memory Mutation Operators?
\end{enumerate}
\item[\emph{RQ3}] How many of the changes in improved HOMs can not be achived by patch-based GI?
\item[\emph{RQ4}] Do Deep-Parameter-optimised library provide further improvement?
\end{enumerate}

\section{Experiment}
\label{sec_exp}

\section{Result}
\label{sec_result}

\section{Threats to Validity}
\label{sec_threat}

\section{Related Work}
\label{sec_related}

\section{Conclusion}
\label{sec_conclusion}

\bibliographystyle{splncs03}
\bibliography{ref.bib}   

\end{document}
% end of file template.tex

