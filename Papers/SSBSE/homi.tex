\documentclass[oribibl]{llncs}
\usepackage{llncsdoc}
\usepackage{graphicx}
\usepackage[numbers]{natbib}

\begin{document}

\title{Searching Higher Order Mutation For Software Improvement}

\author{Fan Wu\inst{1}         \and
		Mark Harman\inst{1}        \and
		Yue Jia\inst{1}         \and
		Jens Krinke\inst{1}
}


\institute{Department of Computer Science, UCL, Gower Street, London WC1E 6BT, UK \\
              \email{\{fan.wu.12,mark.harman,yue.jia,j.krinke\}@ucl.ac.uk} 
}

\maketitle

\begin{abstract}
Automation has been applied to software improvement in many ways.
However, it suffers from the dilemma between scalability and granularity.
We propose to use equivalent Higher Order Mutants to improve multiple non-functional properties of a software while the functional behavior is preserved.
We find that: zeller numbers go here.
%\keywords{Genetic Improvement \and SBSE \and Higher Order Mutation}
\end{abstract}

\section{Introduction}
\label{sec_intro}

Optimising softwares for better performance such as speed and memory consumption can be demanding, especially when the resources in the running environment is limited.
Manually optimising the performance while keeping/improving the functional behavior of a software is challenging, 
and even harder if there are non-functional properties competing with each other~\cite{Harman:2012:GCC:2351676.2351678}.
Previous studies have applied Genetic Programming and other techniques to automate this optimisation process~\cite{6035728, geneticimprovementJP, Wu:2015:DPO:2739480.2754648}.


\section{Background}
\label{sec_back}

\section{Method}
\label{sec_method}

\section{Experiment}
\label{sec_exp}

\section{Result}
\label{sec_result}

\section{Threats to Validity}
\label{sec_threat}

\section{Related Work}
\label{sec_related}

\section{Conclusion}
\label{sec_conclusion}

\bibliography{ref.bib}   

\end{document}
% end of file template.tex

